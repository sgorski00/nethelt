\chapter{Wstęp}

\section{Wprowadzenie}
Aktualnie nie sposób wyobrazić sobie firmy, przedsiębiorstwa, a nawet gospodarstwa domowego, w którym nie korzysta się z internetu. Dostęp do zewnętrznych serwisów i usług stał się naszą codziennością. Korzystamy z internetu oraz urządzeń, łączących się do niego każdego dnia. W domach internet służy nam często do rozrywki i ułatwiania codziennych czynności, do wyszukiwania informacji i rozwiązywania problemów. Dzięki niemu jesteśmy w stałym kontakcie z innymi, mamy dostęp do płatności elektronicznych, a zakupy możemy zrobić nie wychodząc z domu. W pracy korzystamy z niego w celu wyszukiwania informacji i przesyłania ich dalej, wysyłania maili, pobierania danych z Państwowych spółek, czy chociażby robienia wideokonferencji z klientami z całego Świata.

Poza dostępem do zewnętrznych usług coraz większą popularność zyskują urządzenia Smart oraz IoT (Internet of Things). Wielu z nas nawet nie ma pojęcia na temat tego jak wiele takich urządzeń znajduje się w naszym otoczeniu. Według IoT Analytics pod koniec 2025 r. liczba podłączonych urządzeń IoT na świecie miała osiągnąć 21,1 miliarda. To wzrost o ok. 14\% w stosunku do roku poprzedniego (DO WERYFIKACJI PRZED ODDANIEM). Według prognozy liczba ta będzie tylko rosła i w 2030 roku może sięgnąć 39 miliardów. Tendencję tą można zaobserwować na rysunku \ref{fig:iot-devices} przedstawiający liczbę aktywnych urządzeń typu IoT wraz z prognozą na przyszłe lata.

\begin{figure}[H]
	\centering
	\includegraphics[width=0.8\textwidth]{./images/liczba-urzadzen-iot.png}
	\caption{Liczba urządzeń typu IoT w latach 2020-2035\\Źródło: https://iot-analytics.com/number-connected-iot-devices/}
	\label{fig:iot-devices}
\end{figure}

Pojawienie się tak dużej ilości urządzeń, z których specyfikacji wynika, że wręcz wymagają stałego połączenia z siecią, tworzy nowe problemy i pytania.
Kluczowym pytaniem, które powinniśmy sobie zadać, jest to, czy przy utracie łączności z siecią takiego urządzenia pojawią się poważne komplikacje. Chociaż w przypadku np. lodówki bądź ekspresu do kawy działającym w sieci, brak połączenia nie wyrządzi sporych szkód, tak jednak w przypadku maszyn na hali produkcyjnej lub sygnalizacji świetlnej w centrum miasta, szkody mogą być spore i mogą zagrażać nie tylko sytuacji majątkowej, ale też bezpieczeństwu ludzi.
W związku z powyższym należy monitorować stan takich urządzeń, a w przypadku ich awarii jak najszybciej dążyć do naprawy usterki.

\section{Cel pracy}

\section{Zakres pracy}

\section{Założenia projektowe}